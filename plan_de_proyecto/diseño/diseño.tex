\documentclass[12pt]{article}
\usepackage[utf8]{inputenc}
\usepackage[spanish, mexico]{babel}
\usepackage[T1]{fontenc}
\usepackage{Alegreya}
\usepackage{geometry}
\geometry{
  top = 10mm,
  bottom = 10mm,
  left = 10mm,
  right = 10mm
}

\title{Diseño}
\author{Erik Rangel Limón}
\date{}

\makeatletter
\def\@maketitle{
  \begin{center}
    {\Large\textbf{\@title}}
  \end{center}
}
\makeatletter

\begin{document}

\maketitle

\pagenumbering{gobble}

\section*{Estructura sintáctica}

En principio, la estructura sintáctica podría seguir el estándar \textsc{Iso/Iec 14977} para \textbf{EBNF} (\textit{Extended Backus-Naur Form}), con algunas extensiones adicionales. El objetivo es que los archivos \texttt{.joy} sean legibles y fáciles de entender para los desarrolladores de lenguajes, facilitando así la descripción de gramáticas.

Las principales características definidas en éste estandar son las siguientes:

\begin{itemize}
\item \textit{Definición}

  Con éste se definen nuevas reglas de producción:

\begin{verbatim}
primera regla = ...
segunda regla = ...
\end{verbatim}

\item \textit{Cadena terminal}

  Con ella se definen una cadena terminal dentro de una regla de producción:

\begin{verbatim}
primera regla = "a"
seguda regla = 'b'
\end{verbatim}

\item \textit{Concatenación}

  Permite la secuenciación de distintas reglas
  
\begin{verbatim}
primera regla = "a" , "b"
\end{verbatim}

Esta regla en principio sólo podría producir la cadena ``ab''

\item \textit{Alternación}

  Permite alternativas para una regla de producción.

\begin{verbatim}
 primera regla = "a" | "b" | "c"
\end{verbatim}

Esta regla produce o bien el caracter ``a'', el caracter ``b'' o el caracter ``c''.

\item \textit{Terminación}

  Para indicar el final de la regla de producción

\begin{verbatim}
 primera regla = "a" | "b" | "c" ;
\end{verbatim}

\item \textit{Opción}

  Para indicar que una expresión puede no estar, o aparecer una vez.

\begin{verbatim}
 primera regla = ["a" | "b" | "c"] , "d" ;
\end{verbatim}

Esta regla produce las cadenas ``d'', ``ad'', ``bd'', ``cd''.

\item \textit{Repetición}

  Para indicar que una expresión puede no estar, o estar un número no determinado de veces.

\begin{verbatim}
primera regla = {"a" | "b" | "c"} , "d" ;
\end{verbatim}

Esta regla produce cadenas que comienzan con una cantidad arbitraria de ``a'', ``b'' y ``c'' (incluyendo cero, en cualquier orden) que terminan en ``d''.

\item \textit{Agrupación}

  Para indicar que una expresión se debe considerar como una sóla.

\begin{verbatim}
primera regla = (a | b) , c 
\end{verbatim}

Esta regla produce las cadenas ``ac'' y ``bc''

\item \textit{Excepción}

  Esta regla permite indicar la diferencia entre conjuntos.

\begin{verbatim}
primera regla = (a | b), c
segunda regla = primera regla - "ac"
\end{verbatim}

En este caso la segunda regla sólo puede producir la cadena ``bc''

\item \textit{Secuencias especiales}

  Esta regla permitiría incluir conjuntos predefinidos en el lenguaje, por ejemplo: enteros de 32 bits, enteros de 64 bits, números de punto flotante, cadenas de texto minúsculas, cadenas de texto en mayúsculas, cadenas numéricas, símbolos especiales, etc...

\begin{verbatim}
primera regla = ? int64 ?
              | ? float ?
              | primera regla + primera regla
\end{verbatim}

Esta regla permitiría sumas entre números de punto flotante y enteros de 64 bits

\end{itemize}

\end{document}
