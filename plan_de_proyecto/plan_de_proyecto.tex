\documentclass[11pt]{article}
\usepackage[utf8]{inputenc}
\usepackage[spanish, mexico]{babel}
\usepackage[T1]{fontenc}
\usepackage{Alegreya}
\usepackage{xurl}
\usepackage{geometry}
\geometry{
  top = 10mm,
  bottom = 10mm,
  left = 10mm,
  right = 10mm
}

\title{Plan de Proyecto}
\author{Erik Rangel Limón}
\date{}

\makeatletter
\def\@maketitle{
  \begin{center}
    {\Large\textbf{Plan de Proyecto}}
  \end{center}
}
\makeatletter

\begin{document}

\pagenumbering{gobble}

\maketitle

\section*{Propuesta de Tema de Tesis}

\subsection*{Título Tentativo}

\noindent{\large\textbf{Joy: \textit{Just Optimized Yielder} - Generación Automática de Pruebas para Gramáticas en Haskell}}

\subsection*{Justificación}

El desarrollo de lenguajes de programación es una tarea compleja que requiere herramientas eficientes para probar y validar la corrección de las gramáticas. Las herramientas actuales como \textit{Yagg}\cite{yagg} y \textit{DGL}\cite{dgl} presentan limitaciones para su integración en flujos de trabajo declarativos. \textit{Yagg}, aunque permite la generación aleatoria de expresiones, fue diseñado con un enfoque imperativo y carece de configuraciones avanzadas para la validación semántica de cadenas. \textit{DGL}, por otro lado, ofrece una configuración flexible de pesos en reglas de producción y documentación extensa, pero al ser un programa externo, su integración en entornos declarativos se ve afectada por la constante comunicación con el sistema operativo.

Este proyecto propone la herramienta \textbf{Joy: \textit{Just Optimized Yielder}}, la cual permite a los desarrolladores definir gramáticas de manera declarativa usando \textit{EBNF} (\textit{Extended Backus-Naur Form}) y generar automáticamente herramientas de prueba para el desarrollo de un lenguaje de programación en \textit{Haskell}. Al utilizar un enfoque monádico\footnote[1]{Las mónadas son una estructura algebráica utilizada para manejar secuencias de operaciones que involucran contextos computacionales o efectos secundarios, como manejo de errores, estado, entrada/salida, o cálculos no determinísticos. En este proyecto principalmente se utilizaría para tener control del estado del programa durante el análisis sintáctico, así como describir la manipulación del flujo de la información generada aleatoriamente.} para el análisis sintáctico y la generación de pruebas, \textbf{Joy} ofrece una forma eficiente y modular de validar lenguajes de programación. Además, al integrar generadores basados en \textit{QuickCheck}\cite{quickcheck_manual}, se garantiza que las pruebas sean robustas y cubran una amplia gama de casos.

En el ámbito académico, Joy pretende contribuir al estudio de lenguajes formales y programación funcional, integrando técnicas como analizadores sintácticos monádicos y generadores de pruebas basados en propiedades (\textit{QuickCheck}), fomentando la investigación en métodos eficientes para el análisis sintáctico y la generación automática de pruebas. Y en el ámbito práctico, Joy pretende simplificar el desarrollo de lenguajes de programación al automatizar la creación de herramientas de validación, reduciendo el tiempo y esfuerzo requeridos para probar gramáticas y mejorando la calidad de los lenguajes mediante pruebas robustas. Su enfoque declarativo y modular lo hace adaptable a diversos contextos, desde lenguajes de dominio específico hasta lenguajes de propósito general, posicionándolo como una solución versátil y valiosa tanto para la industria como para la academia.

\subsection*{Objetivos}

\subsubsection*{Objetivo General}

Desarrollar una herramienta llamada \textbf{Joy: \textit{Just Optimized Yielder}} que permita a los desarrolladores de lenguajes de programación generar automáticamente casos de prueba para validar gramáticas en formato \textit{EBNF}.

\subsubsection*{Objetivos Específicos}

\begin{enumerate}
\item Desarrollar un analizador sintáctico en \textit{Haskell} para integrar archivos \texttt{.joy} con gramáticas en formato \textit{EBNF}, proporcionando una estructura comprensible y utilizable como documentación del lenguaje.
\item Generar automáticamente código de \textit{Haskell} que incluya generadores de pruebas para cada regla de producción definida en la gramática. Estos generadores seguirían un enfoque basado en propiedades.
\item Incorporar \textit{QuickCheck} como base para la generación de pruebas basadas en propiedades, asegurando una validación exhaustiva de las gramáticas analizadas.
\item Permitir la configuración de parámetros adicionales para restringir tipos de datos, determinar las reglas de producción a usar en los generadores, o determinar el tamaño de la longitud del árbol.
\item Opcional: Implementar la capacidad de evaluar cadenas generadas mediante un intérprete configurable.
\item Documentar y validar la herramienta mediante casos de estudio concretos.
\end{enumerate}

\section*{Descripción de los casos de estudio}

Para validar la efectividad de \textit{Joy}, se proponen los siguientes casos de estudio:

\begin{enumerate}
\item \textit{Lenguaje de Expresiones Aritméticas}

  \begin{itemize}
  \item \textit{Descripción:} Un lenguaje simple que incluye operaciones básicas como suma, resta, multiplicación y división.
  \item \textit{Objetivos:} Evaluar que los generadores de pruebas produzcan una cobertura representativa de las expresiones aritméticas permitidas por la gramática.
  \end{itemize}
  
\item \textit{Expresiones básicas en LISP}

  \begin{itemize}
  \item \textit{Descripción:} Un subconjunto del lenguaje \textit{LISP} con funcionalidades simples como la definición de variables mediante expresiones \texttt{let} y la aplicación de funciones.
  \item \textit{Objetivo:} Validar que los generadores produzcan expresiones correctas tanto sintácticamente como semánticamente con ayuda de un intérprete externo, como \textit{Racket}.
  \end{itemize}

\item \textit{Programación imperativa}

  \begin{itemize}
  \item \textit{Descripción:} Un lenguaje completo, que cuente con la capacidad de generar procedimientos descritos de forma imperativa, expresiones \texttt{if}, ciclos \texttt{while}, \texttt{for}, definición de métodos, etc\ldots
  \item \textit{Objetivo:} Comprobar que el sistema desarrollado es capaz de generar expresiones de un lenguaje con más especificaciones, y evaluar la efectividad de los generadores de pruebas en el desarrollo y validación de un lenguaje de programación con características imperativas.
  \end{itemize}
  
\end{enumerate}

\section*{Metodología}

\begin{enumerate}
\item \textit{Investigación y Análisis}:

  \begin{itemize}
  \item Revisar el estado del arte en herramientas de generación de pruebas y análisis sintáctico.
  \item Profundizar sobre bibliotecas de \textit{Haskell} relevantes (\textit{QuickCheck},\textit{State},etc\ldots).
  \item Investigar bibliotecas similares en otros lenguajes.
  \end{itemize}
\item \textit{Diseño}:

  \begin{itemize}
  \item Definir la estructura sintáctica de los archivos \texttt{.joy}, que a diferencia de otros como el del lenguaje de especificación de \textit{Happy}\cite{happy_docs}, seguirá el estandar \textsc{Iso/Iec 14977}\cite{ebnf} de \textit{EBNF} para una mejor comprensión de la gramática.
  \item Diseñar el analizador sintáctico basado en \textit{parsers} monádicos desde cero.
  \item Diseñar los generadores de pruebas y su integración con \textit{QuickCheck}
  \end{itemize}
\item \textit{Implementación}:

  \begin{itemize}
  \item Desarrollar el analizador sintáctico en \textit{Haskell}.
  \item Implementar la traducción automática de \textit{EBNF} a generadores aleatorios en \textit{QuickCheck}.
  \item Integrar la especificación de parámetros adicionales para restricciones en los generadores, como longitud de las cadenas generadas, exclusión de reglas de producción, pesos que determinen la frecuencia de las reglas de producción, entre otras.
  \item Implementar, de manera opcional, la evaluación mediante intérpretes externos.
  \end{itemize}
\item \textit{Validación}:

  \begin{itemize}
  \item Aplicar los casos de estudio para validar la funcionalidad de \textbf{Joy}.

  \item Realizar pruebas unitarias y de integración por cada componente.

    \begin{itemize}
    \item Comprobar que las expresiones generadas siguen la estructura sintáctica que se definió.
    \item Verificar que las expresiones cumplen con las restricciones que se especificaron al generador.
    \item Corroborar que se crean expresiones semánticamente correctas mediante el uso de intérpretes externos.
      
    \item Probar la extensibilidad de las pruebas y su integración con \textit{QuickCheck}.

    \item Comparar el tiempo de compilación con el de otras herramientas similares (e.g. \textit{Happy}).
    \end{itemize}
  \end{itemize}
\item \textit{Redacción del Trabajo de Tesis}
  
  \begin{itemize}
  \item Redactar de la documentación técnica y del usuario.
  \item Preparar ejemplos y tutoriales.
  \end{itemize}

\item \textit{Evaluación y Mejora}:

  \begin{itemize}
  \item Revisar los resultados.
  \item Identificar áreas de mejora y optimización.
  \end{itemize}
\end{enumerate}

\section*{Resultados Esperados}

Se espera que \textbf{Joy} sea una herramienta funcional y eficiente que cumpla con los siguientes resultados:

\begin{itemize}
\item Generación automática de expresiones a partir de gramáticas descritas en \textit{EBNF}.
\item Integración exitosa con \textit{QuickCheck} para la generación de casos de prueba robustos.
\item Capacidad para manejar gramáticas libres del contexto, como lenguajes de programación de propósito específico.
\item Documentación clara y ejemplos que faciliten su uso por parte de otros desarrolladores.
\item Validación exitosa de cada componente, mediante los casos de estudio propuestos y las pruebas unitarias mencionadas.
\end{itemize}

Asimismo, se espera que \textbf{Joy} sirva como una contribución significativa al campo de desarrollo de lenguajes de programación, promoviendo el uso de técnicas declarativas y funcionales.

\section*{Índice Propuesto}

El documento de la tesis seguirá la siguiente estructura:

\begin{enumerate}
\item \textbf{Introducción}
  
  \begin{itemize}
  \item Contexto y motivación.
  \item Objetivos del proyecto.
  \item Antecedentes: analizadores sintáticos monádicos, proceso del compilador, \textit{QuickCheck} y sus módulos.
  \end{itemize}

\item \textbf{Estado del Arte}

  \begin{itemize}
  \item Herramientas existentes para generación de pruebas.
  \item Análisis sintáctico en \textit{Haskell}.
  \end{itemize}

\item \textbf{Diseño de Joy}

  \begin{itemize}
  \item Descripción de la herramienta.
  \item Especificación del formato \texttt{.joy}
  \item Arquitectura del sistema.
  \end{itemize}

\item \textbf{Implementación}

  \begin{itemize}
  \item Analizador sintáctico.
  \item Generadores de pruebas.
  \item Integración con \textit{QuickCheck}.
  \item Funcionalidades opcionales.
  \end{itemize}
  
\item \textbf{Casos de Estudio}

  \begin{itemize}
  \item Lenguaje de expresiones aritméticas.
  \item Subconjunto del lenguaje \textit{LISP}.
  \item Lenguaje imperativo de propósito específico.
  \end{itemize}

\item \textbf{Resultados y Discusión}

  \begin{itemize}
  \item Evaluación de la herramienta.
  \item Limitaciones y áreas de mejora.
  \end{itemize}

\item \textbf{Conclusiones y Trabajo Futuro}

  \begin{itemize}
  \item Conclusiones del proyecto.
  \item Posibles extensiones y mejoras.
  \end{itemize}
\item \textbf{Referencias}

  \begin{itemize}
  \item Bibliografía y recursos utilizados.
  \end{itemize}

\item \textbf{Anexos}

  \begin{itemize}
  \item Código fuente.
  \item Repositorio con el contenido del proyecto.
  \item Manual de usuario.
  \end{itemize}
\end{enumerate}

\begin{thebibliography}{12}
\bibitem{lyah} 
  Lipovaca, M. (2011). \textit{Learn you a Haskell for great good! A beginner's guide} (1st ed.). No Starch Press.

\bibitem{ph} 
  Hutton, G. (2007). \textit{Programming in Haskell} (3rd ed.). Cambridge University Press.

\bibitem{tcpf} 
  Pedraza, D. (2018). \textit{Teoría de categorías y programación funcional} (Trabajo de titulación). Universidad de Sevilla.
  
\bibitem{fpl}
  Lee, K. D. (2017). \textit{Foundations of programming languages}. Springer.
  
\bibitem{tpl}
  Pierce, B. C. (2002). \textit{Types and programming languages}. MIT Press.

\bibitem{cld} 
  Thain, D. (2020). \textit{Introduction to compilers and language design} (2nd ed.). University of Notre Dame.

\bibitem{ccpp}
  Louden, K. C. (1997). \textit{Compiler construction: Principles and practice}. PWS Publishing Company.

\bibitem{ebnf}
  International Organization for Standardization. (1996). \textit{Information technology — Syntactic metalanguage — Extended BNF} (ISO/IEC 14977:1996). Recuperado de \url{https://www.cl.cam.ac.uk/~mgk25/iso-14977.pdf}

\bibitem{yagg}
  Coppit, D. (2018, julio). \textit{Random generator manual}. Recuperado de \url{https://metacpan.org/dist/yagg/view/random_generator}

\bibitem{dgl}
  Maurer, H. A. (n.d.). \textit{DGL manual}. Recuperado de \url{https://cs.baylor.edu/~maurer/aida/dgl-source/documentation/dglmanul.pdf}

\bibitem{quickcheck_manual}
  Claessen, K., \& Hughes, J. (n.d.). \textit{QuickCheck manual}. Recuperado de \url{https://www.cse.chalmers.se/~rjmh/QuickCheck/manual.html}

\bibitem{happy_docs}
  Happy. (n.d.). \textit{Happy documentation}. Recuperado de \url{https://haskell-happy.readthedocs.io/en/latest/}
  
\end{thebibliography}

\end{document}
